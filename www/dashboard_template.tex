% Options for packages loaded elsewhere
\PassOptionsToPackage{unicode}{hyperref}
\PassOptionsToPackage{hyphens}{url}
%
\documentclass[
]{article}
\usepackage{amsmath,amssymb}
\usepackage{iftex}
\ifPDFTeX
  \usepackage[T1]{fontenc}
  \usepackage[utf8]{inputenc}
  \usepackage{textcomp} % provide euro and other symbols
\else % if luatex or xetex
  \usepackage{unicode-math} % this also loads fontspec
  \defaultfontfeatures{Scale=MatchLowercase}
  \defaultfontfeatures[\rmfamily]{Ligatures=TeX,Scale=1}
\fi
\usepackage{lmodern}
\ifPDFTeX\else
  % xetex/luatex font selection
\fi
% Use upquote if available, for straight quotes in verbatim environments
\IfFileExists{upquote.sty}{\usepackage{upquote}}{}
\IfFileExists{microtype.sty}{% use microtype if available
  \usepackage[]{microtype}
  \UseMicrotypeSet[protrusion]{basicmath} % disable protrusion for tt fonts
}{}
\makeatletter
\@ifundefined{KOMAClassName}{% if non-KOMA class
  \IfFileExists{parskip.sty}{%
    \usepackage{parskip}
  }{% else
    \setlength{\parindent}{0pt}
    \setlength{\parskip}{6pt plus 2pt minus 1pt}}
}{% if KOMA class
  \KOMAoptions{parskip=half}}
\makeatother
\usepackage{xcolor}
\usepackage[margin=1in]{geometry}
\usepackage{graphicx}
\makeatletter
\def\maxwidth{\ifdim\Gin@nat@width>\linewidth\linewidth\else\Gin@nat@width\fi}
\def\maxheight{\ifdim\Gin@nat@height>\textheight\textheight\else\Gin@nat@height\fi}
\makeatother
% Scale images if necessary, so that they will not overflow the page
% margins by default, and it is still possible to overwrite the defaults
% using explicit options in \includegraphics[width, height, ...]{}
\setkeys{Gin}{width=\maxwidth,height=\maxheight,keepaspectratio}
% Set default figure placement to htbp
\makeatletter
\def\fps@figure{htbp}
\makeatother
\setlength{\emergencystretch}{3em} % prevent overfull lines
\providecommand{\tightlist}{%
  \setlength{\itemsep}{0pt}\setlength{\parskip}{0pt}}
\setcounter{secnumdepth}{-\maxdimen} % remove section numbering
\usepackage{ragged2e}
\usepackage{caption}
\usepackage{booktabs}
\ifLuaTeX
  \usepackage{selnolig}  % disable illegal ligatures
\fi
\usepackage{bookmark}
\IfFileExists{xurl.sty}{\usepackage{xurl}}{} % add URL line breaks if available
\urlstyle{same}
\hypersetup{
  pdftitle={Analyse du département Non spécifié},
  pdfauthor={Pierre QUINTIN de KERCADIO},
  hidelinks,
  pdfcreator={LaTeX via pandoc}}

\title{Analyse du département Non spécifié}
\author{Pierre QUINTIN de KERCADIO}
\date{}

\begin{document}
\maketitle

\section{Introduction}\label{introduction}

Ce rapport présente une analyse socio-économique du département
\textbf{Non spécifié} basée sur les données de 2022. Il présente
plusieurs indicateurs clés : taux de chômage, revenu moyen, l'indice de
transport, l'indice construction et croissance démographique. Il inclut
une carte situant le département, un tableau récapitulatif de ses
indicateurs, ainsi qu'une comparaison avec la moyenne nationale.'')

\section{\texorpdfstring{\textbf{Carte de France avec le département
sélectionné Non
spécifié}}{Carte de France avec le département sélectionné Non spécifié}}\label{carte-de-france-avec-le-duxe9partement-suxe9lectionnuxe9-non-spuxe9cifiuxe9}

\begin{center}
\includegraphics[width=0.8\textwidth]{}
\end{center}

\newpage

\section{\texorpdfstring{\textbf{Résumé des Indicateurs du département
Non
spécifié}}{Résumé des Indicateurs du département Non spécifié}}\label{ruxe9sumuxe9-des-indicateurs-du-duxe9partement-non-spuxe9cifiuxe9}

\begin{table}[h]
\centering
\renewcommand{\arraystretch}{1.3} % Ajustement de la hauteur des lignes
\begin{tabular}{|l|r|}
\hline
Indicateur & Valeur \\
\hline
Taux de Chômage (\%) & 0 \% \\
Revenu Moyen (€) & 0 € \\
Indice de Transport & 0 \\
Indice de Construction  & 0 \\
Croissance Démographique (\%) & 0 \\
\hline
\end{tabular}
\caption{Indicateurs socio-économiques du département Non spécifié}
\end{table}

\section{\texorpdfstring{\textbf{Moyenne Nationale des
Indicateurs}}{Moyenne Nationale des Indicateurs}}\label{moyenne-nationale-des-indicateurs}

\begin{table}[h]
\centering
\renewcommand{\arraystretch}{1.3}
\begin{tabular}{|l|r|}
\hline
Indicateur & Moyenne Nationale \\
\hline
Taux de Chômage (\%) & 0 \% \\
Revenu Moyen (€) & 0 € \\
Indice de Transport & 0 \\
Indice de Construction  & 0 \\
Croissance Démographique (\%) & 0 \\
\hline
\end{tabular}
\caption{Moyenne nationale des indicateurs socio-économiques}
\end{table}

\end{document}
